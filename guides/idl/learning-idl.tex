\section{The quick way to learn IDL}\label{qs-learning-idl}

I'm a big fan of the ``experimental'' school of learning. I believe you
should just play around with something to learn how it works. Probably
not a good idea with explosives, but that's not what we're dealing with
here. It's pretty hard to make the IDL interpreter
``blow-up''\footnote{Although I have done it}, so there's almost no
worry that you can actually hurt anything. 

Believe me, you'll learn more sitting at the IDL prompt, typing commands
into the interpreter and examining the results by using ``print'' and
``help'' than you'll ever learn reading a book all the way through and
trying to write your first widget program before you've ever read a data
file or made a simple plot.

So, I encourage you to start immediately by trying things. The first
things you need to learn are. (along with a partial list of the \textit{routines}
used in the activity )

\be

  \item Creating, modifying and destroying variables, particularly
  arrays (all the \textit{???arr} family of array creators
  (e.g. \textit{bytarr}, \textit{fltarr}, \ldots, etc),
  \textit{replicate}, \textit{make\_array}, the type casting functions
  (e.g. \textit{byte}, \textit{fix}, \textit{long}, \textit{float},
  \textit{string}, \ldots, etc.)

  \item Getting help/info about variables (\textit{print},\textit{help},\textit{size},\textit{n\_elements})

  \item Indexing into arrays, particularly using slices, index vectors
  and where vectors (the \textit{?indgen} family index generators
  (e.g. \textit{indgen}, \textit{findgen}, \ldots, etc.) and \textit{where})


  \item Reshaping arrays (\textit{transpose}, \textit{rebin}, \textit{reform},
\textit{congrid})

  \item Calling procedures and functions

\ee

  After you've built up some familiarity with these things (which may
  take you as little  as a couple of hours), you can move on to

\be

  \item File I/O (\textit{readf} for text and \textit{readu} for binary, the \textit{HDF\_???}
  routines for HDF data, \textit{CDF\_???} family for NetCDF data,
  \ldots, etc)

  \item 2-d plotting (\textit{plot}, \textit{oplot}, \textit{plots})

  \item 3-d plotting (\textit{surface}, \textit{shade\_surf},
  \textit{contour})

  \item Image plotting and manipulation (\textit{TV},\textit{TVSCL},\textit{TVRD},\textit{TVCLT})

\ee

  And finally you can grauduate to creating your own procedures and
  functions, which will probably lead you quickly to structures. After
  some time, you might encounter pointers, widget programming and,
  finally, object-oriented programming.

  After that, there's nothing left by object graphics (and I haven't
  even gotten this far yet)

  There's a lot to learn. The good thing is that IDL is relatively easy
  to get started using, once you understand some simple ideas.