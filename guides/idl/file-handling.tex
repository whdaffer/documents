\section{Vanilla File Handling}\ref{sec:vanilla-file-handling}

\section{Searching}

I'm only going to give you some of the most useful arguments/keywords
to these routines, some of which have many. To learn more, look up the
routines in the IDL help file under ``IDL Reference.''

\IDL{files=file\_search(``search\_expression'')} 

or 

\IDL{files=file\_search(``directory'',''recursion\_pattern'')}

In the first prototype you pass a file glob, whatever your native OS
supports, like ``/path/to/source/files/*.c'' and IDL returns all the
files that match that glob. You should be able to use any glob which the native OS supports, so for unix that means ``*'', ``?'', character classes like ``[abc]'' and alternatives like ``{a,b,c}''. For example

\IDL{files=file\_search(``/path/to/my/source/files/*.{c,h}'')} 

should find the C source and header files in that oddly named
directory.

In the second form you give a directory (or directories) and a pattern
in the second argument. IDL recurses down the tree from each of the input directories looking for files which match the pattern. So \ldots

\IDL{files=file\_search(``/path/to/my/source/files'', ``*.{c,h}'')} 

would recurse down from ``/path/to/my/source/files'' finding all C
source and header files and returning them in ``files''



\section{Parsing File Names}

\IDL{path=file\_basename(files)}

Returns the file name, minus any paths. (IDL analog to the unix
utility ``basename'')

\IDL{path=file\_dirname(files)}

The obverse of basename, return the ``path'' of each file.

%\IDL{}


\section{Widget Routines}

\IDL{files=dialog\_pickfile(pattern[,path=''path'',/multi,/read,/write])}

Pop up a widget with a file choosing dialog that allows you to
navigate through the directory tree and choose one file or multiple
files (if /multiple)
