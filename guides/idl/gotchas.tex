\section{Gotchas}\label{sec:Gotchas}

  \begin{description}
    \item \textit{Scalar versus 1-element vector}

     There are circumstances where there's no difference between a
     scalar and a one element vector or array.  The usual case is
     addressing a scalar as an array. For instance, say you have a
     scalar string:

     \IDL{foo= ``bar''}
  
     If is perfectly permissable to say:

     \IDL{foo[0] = ``baz''}

     in order to change the value of that scalar string.

     By analogy, you may any number (up to 8) of ``fictious'' trailing
     dimensions, provided you only index their first (i.e. ``0-th'').
     element!   I could just as easily have done

     \IDL{foo[0,0,0,0,0,0,0,0] = ``baz''} 

     It would have had the same result. IDL quietly ignores the fact
     that ``foo'' is a scalar when the address is to a scalar, the
     first of any such fictious array. Put a number different from
     zero in any of those slots and you'll get a syntax error!

     However, this situation is not symetrical in all
     circumstances. For instance, if I do the following:

     \IDL{foo = [``bar'']}\\
     \IDL{IF strlen(foo) gt 1 THEN print, ``Bigger than 1!''}

     I'll get a syntax error complaining:


\begin{alltt}
% Expression must be a scalar 
%   in this context: <BYTE Array[1]>.
% Execution halted at:  $MAIN$                 
\end{alltt}

    Solution:

    In tests (e.g. IF/THEN DO/WHILE \ldots) , always index the
    putative scalar. If the item is a singleton array masquerading as
    a scalar you'll protect yourself against these sorts of errors.

    In functions which return arrays, always test for singleton arrays
    just before the return and ``scalarize'' the singleton array,
    comme ca:

\begin{alltt}
if n_elements(foo) eq 1 then foo=foo[0]
return,foo
\end{alltt}

    \item \textit{Anonymous structures not being ``anonymous''
    enough.}

\end{description}
