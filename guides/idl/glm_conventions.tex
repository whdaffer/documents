\section{Conventions}\label{sec:Conventions}

    Code to be entered at the the IDL command line or examples of same will
    appear as:

    \IDL{foo}
    
    and if there are multiple lines, as

    \begin{IDLExample}
      \begin{verbatim}

       This is the first line

       And this is the second

      \end{verbatim}
    \end{IDLExample}
        

    Examples of IDL code, particular \prit{BLOCK} code will appear
    something like:


    \begin{IDLExample}
\begin{verbatim}

    IF ... THEN BEGIN
      statement 1
    ENDIF ELSE BEGIN
      statement 2
    ENDELSE

     \end{verbatim}
 \end{IDLExample}\index{if/then/else}

\section{Assumptions and Method}\label{sec:Assumptions}

    I assume that the reader understands:
    \begin{itemize}
        \item The fundamentals of software languages: operator
         precedence, truth/falsity, the notion of ``assignment,'' 
         of an \textbf{L-value}, \ldots etc.

        \item The difference between an \textit{interpreted} and
              \textit{compiled} languages.
         \item and the basics of vectors and arrays.
    \end{itemize}

   I further assume that:

   \begin{itemize}

     \item That we are working on a Unix system, using X11.

     \item That your IDL version $>$= 5.2

   \end{itemize}


   I'm going to try to give examples which will be useful to people
  doing our sort of work, but I'm not going to try to define
  everything before I use it. Rather I'm going to assume that the
  reader will be able to get the sense of what I'm doing in my
  examples and go look up everything they need to on their own. If
  this works, by the time you see the explanation of any particular
  used but undefined element of the IDL language you will most likely
  already have guessed at the explanation, and things will move that
  much faster for all of us.
  
  And I will use several IDL routines without really explaining them.
  I'll try to put comments that explain the situation, but I'm
  expecting you to go look up any routines you don't understand. To
  that end, I expect much use of IDL's online help facility.

\newpage
\section{Getting Help}\label{sec:Getting-Help}

  \SH{idlhelp \&}

   -- or --

   \IDL{?}

   Hit the ``index'' tab on the window which opens up. Type in
   whatever is troubling you. Follow your nose.

   You can get help about the variables in your IDL session by doing 

   \IDL{help}\index{Useful Routines!help}\index{getting help}
   
   which will show you a list of (basically) everything going on in
   your session. You can limit the information by using keywords to
   the \textit{help} command. For example, to see help which files are
   open, do

  \IDL{help,/files}

  To get information on your current graphics environment, i.e. where
  the output from the graphics routines will go, do

  \IDL{help,/graphics}

  You should probably use idlhelp to look up the help command right now.





