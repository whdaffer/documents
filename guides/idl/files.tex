\section{Files}\label{sec:Files}

  It isn't long before one needs to write data to and read data from
  some kind of file. This section addresses this need. 

  There are three basic types of files in ascending order of
  complexity. These are:

  \begin{itemize}
     \item \textbf{Text Files}
     \item \textbf{Generic (i.e. flat) Binary Files}
     \item \textbf{HDF/NCDF Files}
  \end{itemize}

  Again, I'm going to assume that we're working on a Unix
  system. Those using Windows, Mac or VMS will have to read
  the online help. Nothing I'm going to write will be wrong for these
  operating systems, but there are other argument that add to the
  functionality for Windows and Mac users.

\subsection{General Commands}\label{sec:GeneralIOCommands}


  Files, whether text, which RSI calls \textit{formatted} data, or
  Generic binary files, which RSI calls \textit{unformatted} data (but
  not the HDF/NCDF type!) files are opened, read and written with the
  same set of commands. One opens them with the \IDLBUILTIN{openX}
  suite of commands. One reads them using \IDLBUILTIN{readX} and
  writes them with either \IDLBUILTIN{printf}, for
  \textit{formatted} data or \IDLBUILTIN{writeu} for
  \textit{unformated} data. The ``X'' varies depending on the
  particular command, X = ``r'', ``w'' or ``u'' in the case of
  ``open'' and X = ``f'' or ``u'' in the case of read. The particulars
  are:

  \begin{itemize}
     \item \textbf{open}

	Prototype: openX, filename, logical\_unit\_number [, \$\\
	   record\_length, \$\\
   	    error=error, /get\_lun \ldots ]

        X = ``r'', ``w'' or ``u''

	\begin{enumerate}
	  \item \textit{R} Read:

	     Open the file for reading. Doesn't matter whether the
             data is text of binary, that element of the operation is
             covered  by which builtin you use to read the data.


	  \item \textit{W} Write:

	     Open the file for writing. If the file doesn't exist
             it's created, if it does it's zeroed out. 

             Again, it doesn't matter whether the data is text of
             binary, that element of the operation is covered by which
             builtin you use to write the data.

	  \item \textit{U} Update: 

	     Open the file for updating. One may both read and write
             to the file. The file pointer is set to the beginning of
             the file.

	\end{enumerate}

	Required arguments:

	\begin{description}

  	  \item \textit{filename} A scalar string, giving the name of
            the file you want to open.

  	  \item \textit{logical\_unit\_number}

	    The ``logical\_unit\_number'', hereinafter LUN, is the
	    logical entity used by IDL to communicate with the
	    file. One can either give a hardcoded LUN, (e.g. openr, 1,
	    file) the \textbf{bad} way to do it, or get a LUN from the
	    store of them maintained by the interpreter using
	    \IDLBUILTIN{get\_lun}. One may also give the /get\_lun
	    keyword in the call to open, which is equivalent to
	    calling \IDLBUILTIN{get\_lun, lun} before the call to
	    \IDLBUILTIN{open}. One returns the LUN back to the pool
	    with \IDLBUILTIN{free\_lun, lun1, lun2, \ldots}. There are
	    only 128 LUNs, so it's not inconceivable that a process
	    can run out of them.

	    If the LUN is allocated via \IDLBUILTIN{get\_lun} then
	    calling \IDLBUILTIN{CLOSE,/all}, in addition to closing
	    all files, will also deallocate all such LUNs.

	\end{description}

	Optional arguments or keywords, but a \textbf{really, really}
	good idea!
 
	\begin{description}
	  \item \textit{/get\_lun}

	   Make open get a LUN from the store and put it into the
     	   named variable ``logical\_unit\_number'' in the argument list.

  	  \item \textit{error = error} 
	   A named variable into which open returns the error code
	   which may result during the open.

	   If one doesn't do this and an error does occur, the default
   	   error catching mechanism takes over. I think it's good
	   practice to try and capture errors and handle them
	   explicitly; you may disagree. 

	   However, if you do use this keyword, you \textit{must}
	   check the value of error immediately after the call to
	   open. This is because \IDLBUILTIN{open} is silent about
	   its failure when the ``error=error'' keyword argumentation
	   is used.

	   error != 0 is \textbf{bad}, The particulars of the error is
	   stored in the system variable !error\_state, which is a
	   structure containing lots of information, in particular a
	   text string gives a little description and it's in
	   !error\_state.msg

           My standard way of opening a file (say for reading) is:


\begin{alltt}
openr, lun, file, /get_lun, error=error
if error ne 0 then begin 
  Message,!error_state.msg
  \vdots
  ;Do something or run-away!
endif 
\end{alltt}



          This sort of approach can be combined with
	   \IDLBUILTIN{catch} to eliminate the if-then-begin block and
	   to get a more generic form of error catching mechanism. See
	   section ~\ref{sec:ErrorHandling} for more information on
	   this approach. For instance, I'll do the following:



\begin{alltt}
catch, error
if error ne 0 then begin 
  Message,!error_state.msg,/continue
  return
endif 
     \vdots

openr, lun, file, /get_lun, error=err
if err ne 0 then Message,!error_state.msg,/noname

     \vdots
;File successfully opened!
    \vdots

\end{alltt}


          In this case, the procedure is exited via the ``return''
	   inside the ``catch'' block, control is passed there after
	   the unsuccessful open by the call to \IDLBUILTIN{message}.

	\end{description}


	

	There are many other keywords to the \IDLBUILTIN{open}
        command; I'll leave you to find out about them.


	Other significant open keywords:

	\begin{description}

	  \item \textit{Append} Places the file pointer is set to the
           end of the file. Normally the file pointer is placed at the
           beginning of the file. If used with \IDLBUILTIN{openw} this
           prevents IDL from truncating the file

          \item \textit{SWAP\_IF\_BIG\_ENDIAN}

          If the machine you're issueing the open command on is Big
          Endian, byteswap multi-byte data as it's being read.

          \item \textit{SWAP\_IF\_LITTLE\_ENDIAN}
	
	  Same as SWAP\_IF\_BIG\_ENDIAN, mutatis mutandis.

	  \item \textit{COMPRESS}

	  Read and write the data in ``gzip'' format. Incompatible
          with /APPEND.

	\end{description}
     \item \textbf{read}
     \item \textbf{print/write}
  \end{itemize}

   

\subsection{Text Files}\label{sec:TextFiles}
  
\subsection{Binary Files}\label{sec:BinaryFiles}
\subsection{HDF Files}\label{sec:HDF Files}
\subsection{NCDF Files}\label{sec:NCDF Files}