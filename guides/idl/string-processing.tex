\section{Vanilla string processing}%%%%\ref{sec:vanilla-string-processing}

In versions prior to IDl 5 (I believe) the only sort of string
processing processing was rudimentary. The basic routines, all of
which work with scalar or array arguments, were:

\bi

   \item strcompress(string[,/remove\_all])

     Compress multiple blanks/white space down to one blank each, or
     remove them all if /remove\_all

   \item strupcase(string)

     Uppercase the string

   \item strlowcase(string)

     Lowercase the string

   \item strtrim(string[,flag])

     Remove leading (flag=1), trailing (flag=0 or flag not present)
     or both (flag=2) blanks.

     This is also a quick way to convert a number to a string if you
     don't care enough to specify a format.

   \item strpos(string,substring
            [,start][,/reverse\_offset]
	    [,/reverse\_search])

     Return the character position at which `substring' occurs
     starting at `start' (if provided) or -1 if not found. Where the
     search starts and which direction it goes depends on the two
     keywords: /reverse\_offset and /reverse\_search. If
     /revserse\_offset then count backwards from the end. If
     /reverse\_search then search backwards instead of forwards.

     Will work on arrays.

   \item strmid(string,first\_character,[,length][,/reverse\_offset])

     Remove 'length' (default : all the charcters it can) characters
     from the string starting at 'first\_character (default = 0
     counting from the beginning or strlen(string)-1 counting from the
     end if /reverse\_offset)

     if /reverse\_offset is set, calculate the offset from the end of
     the string, rather than the beginning.

     Example:

     \IDL{print,strmid('foo.bar',0,3)}

     prints ``foo''

     \IDL{print,strmid('foo.bar',2,3,/reverse)}

      print ``bar''
     
    
    \item strlen(string)

    prints the length of the string


    \item strput, destination, source[, position]

    inserts ``source'' into ``desination'' starting at ``position'' (0
    by default)

    \IDL{t=''foo.bar'' \& strput,t,''baz'',4 \& print,t}
     
     prints ``foo.baz''

    \item string(variable [,format=fmt\_spec])

    Converts ``variable'' to a string. If ``format'' isn't present,
    IDL uses it's default conversion rules. If you want more control,
    you can specify a format, using one of two format specifications,
    a ``fortran'' style and a ``C'' style. This is a complicated
    subject which I'm going to shamelesslly avoid. If you want further
    information, see ``Using Explicitly Formatted Input/Output'' in
    the IDL manual.


\ei

  A newer, slightly improved version of strpos is strmatch, which
  allows for a limited set of wildcards, like those one finds in most
  Unix shells, namely '*' (multiple characters), '?' (a single
  character) and '[...]', for character classes The prototype is:

  strmatch(string,search\_string [,/fold\_case])


\section{Splitting/Joining}%%\ref{sec:split-join}

\bi

  \item strsplit(string,pattern [,count=count]
                [,escape=string][,/regex][,/fold\_case]
                [,/extract][,length=length][,/preserve\_null])

   Split the string using ``pattern''. Return the number of fields in
   ``count'' (if present). Returns the character that starts the next
   field unless /extract is set, in which case it returns the
   extracted fields. Case matters unless /fold\_case is set, the
   lengths of the fields is returned in ``length'' (best used if you
   haven't set /extract). Null results are quietly discarded unless
   /preserve\_null is set. This means that

  \IDL{aa=strsplit(``a,b,,c,d'','','',/extract)}

   returns [``a'',''b'',''c'',''d''] 

   while

    \IDL{aa=strsplit(``a,b,,c,d'','','',/extract,/preserve\_null)}

  returns [``a'',''b'','''',''c'',''d''] 

  If /regex, then ``pattern'' is a regular expression, not a simple
  string. More on this later.

  If not doing regex splitting you can use ``escape'' to remove from
  consideration what would otherwise count as a delimiter to split
  on. For example

  \IDL{a=strsplit(``a,b,c'', ``,'')} 

  splits the string into three fields, ``a'',''b'' and ``c''. However,

  \IDL{a=strsplit(``a\,b,c'', ``,'', escape=''\") }

    splits it into two fields, ``a,b'' and ``c''.

   
    /escape and /regex are incompatible.


  \item strjoin(array,delim [,/single])

\ei


\section{Regular Expression string manipulation}%%\ref{sec:Regex}

  Regular expression searching was introduced in IDL 5.4 (I think) and
  is a tremendous help for those tired of having to slog through
  string processing the old fashioned way. T

  Alas, the downside of regular expression is the fact that you have
  to learn about regular expressions.

  I'm not going to go discuss this topic in full, mainly because there
  are whole books written on the subject. I'll give you the prototype,
  point you to the page in the IDL documentation that discusses this
  and give you a few examples.

\bi
  \item strregex(string,regex[,/boolean][,/extract]
          [,length=length][,/subexp][,/fold\_case] )

   Returns an array of items. What is returned depends on which set of
  keywords is set. If /boolean, then it returns 1 whenever the regular
  expression matches, 0 otherwise. Set /fold\_case for case
  insensitive matching. If you don't set /extract, it returns the
  start of the match with the length returned in ``length.'' (if
  present) With /extract, it returns the match or matches (there can
  be more than one, depending on the regular expression), if anything
  matched, or the empty string if it didn't. If you don't use
  sub-expressions, or if you do but you don't set /subexp, it returns
  the part of each string that matches the regular expression, the
  empty string if nothing matches for that string. 

  Before I go into the meaning of the /subexp keyword, let me discuss
  regular expressions in general. 

  

\ei

  \subsection{Regular Expressions}%%\ref{sec:regular expressions}\index{regular expressions}

  The best way to think of regular expressions is not as string
  searches, \prit{per se} but as little programs in an incredibly
  condensed, arcane language. This language allows one to specify
  strings of varying length either exactly or with a multitude of
  variations, including sub-strings of 0 or more length occuring 0 or
  more times. In addition, one may capture substrings and return them
  as sub-expressions (the /subexp above)

  The semantics are:

  \bi

    \item the basis Regexs
     \bi
        \item ``.'' Matches any character, except the newline

	  ``a..b'' matches any four character string starting with an
  ``a'' and ending with a ``b''.

	\item any character that doesn't have a special meaning
  matches itself, i.e. x matches x, y matches y, 1 matches 1, and so
  on.

		the regular expression ``abc'' matches itself.
  
       \item ``(\ldots)'' Grouping regular expressions

       \item ``[\ldots]'' Character class, matches any of a group of
  characters
        

	\item ``a$|$b'' Alternation: constructing matches of one of two
         (or more) alternative regexs


       \item ``\BS'', the ``escape'' character. Putting a backslash
  before a character that has some special meaning to the regular
  expression processor removes that special meaning, e.g. ``\BS .''  is
  the regular expression matching a single dot, so the backslash has
  removed the special meaning of ``.''

     \ei

    \item Positional Anchors
  
     \bi
  
       \item \caret Anchors the regex to the beginning of a line

	 So the regex ``\caret Call me Ishmael'' matches the beginning
  of \prbf{Moby Dick}, but it doesn't match the beginning of an essay
  about that book, which begins ``Herman Melville's great novel
  \prbf{Moby Dick} begins ``Call me Ishmael.'''', since that phrase
  doesn't occur at the beginning of the line.

       \item \$ Anchors the regex to the end of a line

        Simile for \$. The regular expression which ends with \$ only
  matches the end of the line.

   Which means, that a handy way to match empty lines is with
  ``\caret\$'' (but only if they have no whitespace. To handle that,
  see \ldots

  \ei 

  \item Character classes

      A list of characters surrounded by square brackets is a
  character-class. This regular expression will match any of the
  characters included in the list however many times allowed for by
  any quantifiers that apply to that regex.

    For example, the characer class [abcd]+ matches one or more
  strings consisting only of the characters ``a'', ``b'', ``c'' and
  ``d''. 

    One may specify some classes as ranges, ``[a-z]'' matches any
  single lowercase character while ``[A-Z]'' a single uppercase
  character. The character class ``[a-zA-Z0-9]+'' matches one or more
  characters and numbers, the character matching regardless of case.

  A regular expression for a valid IDL variable name could be

  ``[a-zA-Z][a-zA-Z0-9\_]*''

   This matches any string which begins with a letter followed by
  zero or more letters (upper or lower case), numbers or underscores.

  As mentioned above, to match purely empty lines, a not infrequent
  task, use ``\caret\$.''

  To match empty lines, which might have some whitespace characters
  like spaces or tabs, use ``\caret [ `` + string(9b) + ``]*\$''
  (remember, we're speaking IDL here: string(9b) is the ascii tab
  character)\footnote{Of course, you could also do remove all the
  whitespace with strcompress(,/remove\_all) and then match on
  ``\caret\$''}


  As you may have noticed, some characters have a special meaning
  inside character classes. The dash ``--'' is a sort of ``range''
  operator, for instance. But what if you want to have a dash as one of
  the characters in the class? And what about the other special
  characters, ``*'', ``\$'', ``\caret''. The rules that cover this
  circumstance are:

\bi

  \item Inside a character class the special characters like ``*'',
  ``\caret'', ``\$'' lose their special meanings (with a few
  exceptions.)

  \item To include a character that has a special meaning for the
  character class construct, put it at the beginning of the class (the
  exception is ``\caret'') The class ``[-abcd]+'' matches any number
  of characters from the class ``[a-d]'' but also any number of
  dashes. To match an open square bracket, put that at the front of
  the class. To match a close square bracket, escape it with ``\\\."

  \item ``\caret'' at the
  beginning of the class ``negates'' the class. The match is not to
  what's in the character class but what isn't. So ``[\caret0-9]+''
  matches everything \prbf{except numbers}. 


  \item to match the ``\caret'', put it anyplace other than the first
  position: the character class ``[a-z0-9\_\caret ]'' matches any
  single lowercase letter, or any single number, an underscore or the
  caret.

\ei

    \item Quantifiers

      
     \bi
        \item ? Zero or one of the previous regex
        \item * Any number, including 0, of the previous regex
        \item + One or more
	\item $\{$M,N$\}$ At least M but no more than N
     \ei

  The regex ``a*'' matches any number of the letter ``a'', including
  zero, hence the regular expression ``ab*c'' matches ``abc'' and
  ``abbc'' and ``abbbbbbbbbbbbbbbbbc'', but it also matches
  ``ac.'' 

  The regex ``ab?c'' matches ``ac'' and ``abc'', but not ``abbc.'' 

  The regex ``ab$\{$2,3$\}$c'' matches ``abbc'' and ``abbbc'', but not
  ``abc'' or ``abbbbc.''

  Putting parentheses around expressions controls the scope of the
  quantifier. For example ``[a-d](bc)+'' matches ``abc'' and ``abcbc''
  and even ``abcbcbcbcbcbcbcbc'', but not ``abbc'' since the
  quantifier applies to the regex ``bc'', not ``b.'' Also note that it
  doesn't match ``a', since the ``+'' requires at least one ``bc.''

    \item Subexpressions

  Aside from controlling the scope of quantifiers, parentheses allow
  for the extraction of subexpressions through the joint use of
  /extract and /subex, or by extacting them yourself using the
  returned array and the variable returned via length=length. Which
  slot in the returned value goes with which sub-expression is
  determined by counting open parentheses starting at the left hand
  side of the regex. The sub-expression enclosed by the first set of
  parens goes into the array at index = 1 (not 0! that, for reasons
  I've never fully understood, is reserved for the whole string) The
  second goes to index=2, and so on.

  This really only gets to be a problem if you use nested subexpressions. 

  My suggestion is that you try it out a few times.



\item Grouping and Alternation

  As mentioned above, one can control how the quantifier works by
  grouping regexs with parentheses. One can also use them with the
  alternation operator ``\VB'' to specify alternatives.

  Let me give you some examples.

  Say you had a filename with a fairly stable naming convention which
  you wanted to find using some sort of file searching but it had one
  part which was 4 characters in one version of the file but 4 numbers
  in another and you wanted to find both without doing two
  searches. You could use the regular expression

  ``.*[a-z]$\{$4$\}|$[0-9]$\{$4$\}$.*'' 

  or you could use 

  ``.*([a-z]$|$[0-9])$\{$4$\}$.*''

  Either should work but the latter version will save whichever was in
  this part of the filename into a subexpression which you can
  retreive later. Which is a nice way to seque back into \ldots





  \ei

  \subsection{Using /extract and /subexp in stregex}

  These are two of the most powerful and useful keywords to this IDL
  routine, the other being /boolean (for testing for conditions)


  When you use /extract and /subexp IDL returns an N by M array, where
  N is the number of subexpressions +1 (IDL puts an implicit set of
  parentheses around the whole regex so that the first subexpression
  returned is the whole match) and M is the number of strings in the
  input array. If ``fields'' is the returned array, then

  fields[1,0] is the first sub-expression match of the first string
  fields[N-1,0] is the last
  fields[1,M-1] is the first sub-expression match of the last string 
  fields[N-1,M-1] is the last.


  Let's do some examples. Say you want to part a time string from a
  file that has the yyyy-mm-ddThh:mm:ss.ccc format, and you want to
  parse out all of the individual elements of the string so that you
  can convert it to julian day and do arithmetic on the times. It
  might go something like this.

  Define an array of times.

\IDL{date=['2004-01-02T00:00:00.000','2004-02-03T01:02:03.456']}

Define a regular expression

  \IDL{regex1='([0-9]+)-([0-9]+)-([0-9]+)T([0-9]+):([0-9]+):([0-9.]+)'}


This says: match anything which has 3 multi-digit numbers separated by a dash, followed by a ``T'', followed by three multi-digit numbers separated by a ``:'', the last of which looks like a float because it has a dot in it.

Now, this regex would certainly match our date strings, but it would
  match other things too. In fact it would match the string
  ``1-2-3T4:5:6.''. 

  You could be more exact and eliminiate some false positives with 


  \IDL{regex1='([0-9]\{4\})-([0-9]\{2\})-([0-9]\{2\})T([0-9]\{2\}):([0-9]\{2\}):([0-9]\{2\}\BS .?[0-9]\{0,3\})'}

  This one says, match 3 sets of numbers separated by dashes, the
  first having 4 digits and the next two having 2, followed by a
  ``T'', followed by three sets of numbers separated by ``:'' , the
  first two having two digits and the last having 2 digits before a
  (possibly absent) the decimal sign, followed by between 0 and three
  digits. 

  It's better, but it could match a string which has five digits in a
  row at the very end (and you see why?)


Better still is:

  \IDL{regex1='([0-9]\{4\})-([0-1][0-9])-([0-3][0-9])T([0-2][0-9]):([0-5][0-9]):([0-6][0-9]\BS .?[0-9]\{0,3\})'}\footnote{Yes, that's a 60 there. Some schema allow it to handle leap-seconds}

But even this would match ``2005-19-39T29:00:69999''

Personally, I'd use the first one when I'm pretty sure the input data
doesn't have any false-positives in it.

Do the extraction

\IDL{t=stregex(date,regex1,/sub,/ex)}

Show what it looks like.

\IDL{ help,t}

prints 

T               STRING    = Array[7, 2]

\IDL{ print,t}

prints \ldots

2004-01-02T00:00:00.000  2004  01  02 00 00 00.000

2004-02-03T01:02:03.456 2004 02 03 01 02 03.456

  The first field in each line is the entire string, the last 6 are
  the subexpressions and correspond to each of the parentheses of my
  regular expression.


\begin{center}
\begin{tabular}{||c|c|c|c|c|c|c||}\hline\hline
Whole Rexex             & 1st paren & 2nd & 3rd & 4th & 5th & 6th\\\hline\hline
2004-01-02T00:00:00.000 & 2004      & 01  & 02  & 00  & 00  & 00.000\\
2004-02-03T01:02:03.456 & 2004      & 02  & 03  & 01  & 02  & 03.456\\\hline
\end{tabular}
\end{center}



