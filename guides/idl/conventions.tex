\section{Conventions}\label{sec:Conventions}

    Code to be entered at the the IDL command line or examples of same will
    appear as:

    \IDL{foo}

    Output from IDL will a appear as 

    \IDLOUT{bar} 

    if there is only one line, and as:


     \begin{alltt}

       \textit{This is the first line}

       \textit{And this is the second}

    \end{alltt}

    if there is more than one or if there are special characters.


    Code to be entered at shell command line will appear as:\\

    \SH{foo}\\

    Examples of IDL code, particular \prit{BLOCK} code will appear
    something like:


\begin{alltt}

    \textbf{IF \ldots THEN BEGIN}\index{if/then/else}
      \textit{foo bar}
    \textbf{ENDIF ELSE BEGIN}
      \textit{ baz }
    \textbf{ENDELSE}

\end{alltt}

\section{Assumptions and Method}\label{sec:Assumptions}

    I assume that the reader understands:
    \begin{itemize}
        \item The fundamentals of software languages: operator
         precedence, truth/falsity, the notion of ``assignment,'' 
         of an \textbf{L-value}, \ldots etc.

        \item The difference between an \textit{interpreted} and
              \textit{compiled} languages.
         \item and the basics of vectors and arrays.
    \end{itemize}

   I further assume that:

   \begin{itemize}

     \item That we are working on a Unix system, using X11.

     \item and when we do graphics, it will be true-color
       (i.e. 32-bit) graphics, but in a manner which will make it
       appear to be 8-bit.

     \item That your IDL version $>$= 5.2
   \end{itemize}


   I'm going to try to give examples which will be useful to people
  doing our sort of work, but I'm not going to try to define
  everything before I use it. Rather I'm going to assume that the
  reader will be able to get the sense of what I'm doing in my
  examples and go look up everything they need to on their own. If
  this works, by the time you see the explanation of any particular
  used but undefined element of the IDL language you will most likely
  already have guessed at the explanation, and things will move that
  much faster for all of us.

  And I will use several IDL routines without really explaining
  them. I'll try to put comments that explain the situation, but I'm
  expecting you to go look up any routines you don't understand.

  Which brings me to my first topic \ldots


\newpage
\section{Getting Help}\label{sec:Getting-Help}

  \SH{idlhelp \&}

   -- or --

   \IDL{?}

   Hit the ``index'' tab on the window which opens up. Type in
   whatever is troubling you. Follow your nose.

   You can get help about the variables in your IDL session by doing 

   \IDL{help}\index{Useful Routines!help}\index{getting help}

  which will show you a list of (basically) everything going on in your
  session. You can limit the information by using keywords to the
  \textit{help} command. For example, to see help just on the variables
  you have defined in your current session, do 

  \IDL{help,/variables}

  or, simply, 

  \IDL{help,/var}

  (IDL allows you to use the smallest unique abbreviation for
  keywords, which is what ``/variables'' is)

  To see which files are open, do

  \IDL{help,/files}

  To get information on your current graphics environment, i.e. where
  the output from the graphics routines will go, do

  \IDL{help,/graphics}

  You should probably use idl help to look up the help command right
  now because there are many, many keywords.





