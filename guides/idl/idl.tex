\input{/Users/whd/latex/mydefs.tex}
%% colors
%%
\newcommand{\IDL}[1]{\textbf{IDL$>$ #1}}
\newcommand{\IDLOUT}[1]{\textsl{#1 }}
\newcommand{\SH}[1]{\textsf{\% #1}}
\newcommand{\emacs}[1]{\textbf{emacs:}\textsl{#1}}
\newcommand{\T}{\textit{T\  }}
\newcommand{\E}{\textit{E\  }}
%\newcommand{\TI}[1]{\textit{T#1\  }}
%\newcommand{\EI}[1]{\textit{E#1\  }}
\newcommand{\IDLBUILTIN}[1]{\textit{#1}}
%
%
\newenvironment{idloutmulti}%
{\begin{minipage}[c][\height][c]{3.0in} \textit}%
{\end{minipage}}
%
%
%
 \NC{\code}{}%
%
% \newlength{\ico} %hoffset for beginning of IDL code sections
% \newlength{\cbo} %hoffset for block
% \newlength{\dho} %default offset
% \setlength{\ico}{0.25in}%
% \setlength{\dho}{\hoffset}
%
% \newenvironment{idlcode}%
% {\begin{code}\setlength{\hoffset}{\ico}\textit}%
% {\end{code}}
% %
% %
% \newenvironment{codeblock}%
% {\begin{code}\setlength{\hoffset}{\hoffset+0.25in}\textit}%
% {\end{code}}%
% %
%
%\NC{\bbm}{\begin{boxedminipage}[t]{4.5}}
%\NC{\ebm}{\end{boxedminipage}}

\NC{\BIC}{\bi}
\NC{\EIC}{\ei}
\NC{\ICI}[1]{\item [] \prit{#1}}

\newenvironment{codebox}%
{\begin{boxedminipage}[t]{4.5in}}
{\end{boxedminipage}}
%

\NC{\BCB}{\bi}
\NC{\ECB}{\ei}
\NC{\CBI}[1]{\item [] \prsl{#1}}


%
%
%
%\newcommand{\IDLCODE}[1]{\begin{idlcode}{#1}\end{idlcode}}
%\newcommand{\IDLOUTMULTI}[1]{\begin{idloutmulti}{#1}\end{idloutmulti}}
%%\newcommand{\IEXP}{\symbol{94}}




\documentclass{article}
\title{Rank Beginner's Guide to using IDL}
\author{W. H. Daffer}
\date{\today}
\begin{document}
\maketitle

\section{Conventions and Assumptions}

  The following assumptions about the users knowledge are made:

\bi
  \item The distinctions between an ``interpreted'' language and one
        that is ``compiled.'' 
  \item The notions of ``subroutine''\footnote{In IDL, this is called
        a ``procedure.''} and ``function'' and the attendant notion of
        a ``calling stack''
  \item The fact that in an interpreter such as IDL the interpreter
        itself is the ``main'' program, and calls all other
        ``procedures'' and ``functions.''
  \item Some basic familiarity with matrix operations.
\ei


 IDL commands to be issued at the command line will appear as:

  \IDL{foo}

  One line text output from the IDL interpreter will appear as:

  \IDLOUT{bar baz}

  Multi-line output will appear as:

  \begin{verbatim}
	A line of output

	Another line of output.

  \end{verbatim}


  Shell Commands to be issued at the Shell command line will appear as:
 
   \SH{foo}


\newpage
\section{Quick Start}

  If you think you you've got the right stuff!

  Examples:

\bi
  \item Creating constants
   
     A byte:

      \IDL{A = 0b}

     A 16 bit (short) integer

       \IDL{A = 0}

     A Longword (4 byte integer)

      \IDL{A = 1L}    

     A float (4 byte)

      \IDL{A = 0.0}
  
     A double (8 byte float)

      \IDL{A = 0.0d}

     A single precision complex number

      \IDL{A = complex(0.0,0.0)}

     A string

     \IDL{A = ``foo''}

     A structure:

     \IDL{a = \LB a:0, b:0.0, c:``a'' \RB }


  \item Creating arrays:
  
     \bi
	\item Explicitly
	  
	   \IDL{A = [0,1,2,3]}
	
	    A is an integer (16 bit) array
    
	   \IDL{A = [0,1,2,3l]}

            A is a longword array (why?)


            \IDL{A = [``a'', ``b'', ``c'']}

	    A is a string array


	    \IDL{A = [ 0b, 1l, 2.0] }
	
	    A is a \prbf{float} array (why?)

	
	\item by call to function

	    \IDL{A=findgen(100)}

		A = [0.0,1.0,2.0, \ldots , 99.0 ]

      	    \IDL{A = bindgen(512) }

	        A = [0b,1,2, \ldots, 255, 0, 1, \ldots, 255]

	        (Get it? It's a \prbf{byte}, so it only goes up to
                255, then in wraps mod 256!)

	    \IDL{A=indgen(2,3)}

	    A is a 2 by 3 array arrange as:

%	\begin{center}
	\[ A = \left[ \begin{array}{ccc}
            0 & 1\\
            2 & 3\\
            4 & 5\\
	\end{array} \right] \]
%	\end{center}

        \item As result of operation

	   \IDL{A = sin(findgen(100)/99*!pi)}

	   A is a one hundred element vector, with a 1 period unit
           amplitude  sine wave in it.

	  \IDL{A = findgen(360)\#replicate(1.,180)}

	  A is a 360 by 180 array, useful for representing longitudes
          on a 1 by 1 degree grid. ``Replicate'' creates an array of
          specified dimension(s) filled with the number appearing as
          the first parameter. In this case, a 180 element vector
          filled with ``1''.

          The ``\#'' is one of the two IDL matrix multipliers, the
          other being ``\#\#'', of which I will have more to say
          later.\footnote{And boy, won't you be confused then!}

	  \IDL{B = replicate(1.,360)\#(findgen(180)-90)}

   	  B is the array of latitudes which go with ``A''

          Note that ``replicate'' and ``findgen'' have been
          switched. You'd suspect that whichever way the longitudes in
          ``A'' ran, the latitudes in ``B'' would run the other way,
          and you'd be right. In fact, the longitudes in A run along
          the rows of the array and the latitudes in B run along
          the columns.
          
     \ei

  \item Types of routines
	
    There are two types of routines, builtin and .pro files. Builtin
    routines are written in C or Fortran and compiled or dynamically
    linked\footnote{see \prit{linkimage} and \prit{call\_external} for
    discussion of how a user may accomplish the same thing themselves}
    into the IDL executable.

    .PRO files are in the IDL ``source'' language. They must be
    ``interpreted'' by the interpreter before being executed. The IDL
    interpreter precompiles these into a pseudo-code, and it is this
    pseudo-code which is actually run. The interpreter finds these
    files by means information contained in the user's environment

  \item Environment

     IDL depends on several environmental variables, IDL\_PATH and
     IDL\_STARTUP. In order for these to work, they must be defined
     before entering IDL.

 
   \bi 
    \item IDL\_PATH

     IDL\_PATH is the standard unix path variable, except that it
     tells the IDL interpreter to find the IDL files. There are,
     however, two differences.
	
     \be 
       \item The Plus sign  (``+'') semantics

         If you put a plus sign ``+'' in front of an element of this
         path, the interpreter will recursively descend starting
         from that directory, adding directories as it goes. This
         means that if you keep all your code in subdirectories of your ~/idl
         directory, you need only at ``+/your/home/dir/idl'' to the
         IDL\_PATH and the interpreter will do the rest for you.

	Example:
	
      \item \LT IDL\_DEFAULT \GT semantics

	You should alway end your path with \LT IDL\_DEFAULT \GT\.
        This is the path to RSI supplied .pro file. 

     \ee

     Example: 

       setenv IDL\_PATH ``+/home/whdaffer/idl:\LT IDL\_DEFAULT \GT''

     This will find all the .pro files in any directory below and
     including /home/whdaffer/idl and all the default .pro files which
     RSI provides.
  
   \item IDL\_STARTUP

     IDL\_STARTUP should point to file which has configuration
     commands in the IDL language to configure the user environment. It
     is run each time IDL starts, so have a care which commands you put
     in there.


     A natural set of commands to put into IDL\_STARTUP is commands
     which configure the graphics environment. Until you become more
     familiar with the ins-and-outs of configuring the graphics
     environment, put the following in there.

    \begin{verbatim}
       device,pseudo=8
       window,colors=256,/free,/pixmap,xsize=10,ysize=10
       wdelete,!d.window
    \end{verbatim}

    This will set your graphics environment to use 8 bit color with
    256 colors.

     If you want to do 24 bit color, you have to remove these lines from
     your IDL\_STARTUP file!

  \ei

  \item A word about Column/Row Major and the arrangement of arrays.

   Forget about column/row major! The only important issue is which
   index varies first. In IDL it goes left to right (like
   Fortran). The first index varies fastest, followed by the second,
   followed by the third, \ldots , up to the eighth.

   Now when you plot an array to the screen, say you're plotting an
   image array, the question of which index is which becomes important
   because we naturally think of a computer screen as being arranged
   by row and column. In that circumstance all you need to know is
   that the IDL array thinks of the first index as the column and the
   second as the row and that this is the natural way to think of it
   in an image processing context.

   But when you go to do linear algebra, you'll find yourself
   reversed, since almost everyone thinks of arrays as row, then
   column. And the ``\#'' operator works backwards from the normal
   linear algebra matrix multiplier, it multiplies the rows of the
   first array against the columns of the second\footnote{although the
   ``\#\#'' works as expected} so have a care with that one.

   Knowing how the data is arranged in memory will solve \prbf{lots}
   of problems for you, since it allows you to read files in one
   slurp, go back and forth between 1-d  and 2-d indexing operations,
   turn really gnarley multiplications involving lots of loops into
   one line pieces of elegance which will astound and amaze your
   compatriots.

   So, start being aware of these matter \prbf{now}

  \item Plotting (and other graphics)

   There are two types of graphics systems in IDL, ``direct'' and
  ``object.'' I won't be discussing ``object'' graphics at all, so forget
  they exist.

  There are four broad categories in ``direct'' graphics (hereinafter
  simply ``graphics''), 2-d plots, 3-d plots, geographic mapping and
  images.

  A word about the first three and about coordinate transforms.

  When you execute any direct graphics command, except TV where
  you're just turning pixels on and off and there's no mapping from
  data to device space, the IDL graphics subsystem must calculate
  which device pixels to turn on given the coordinates you've passed
  in the command. Some commands presuppose the existence of this
  corrdinate transform while others create them anew. If you get the
  two broad categories mixed up, you'll be seeing a lot of error
  messages.

  Some commands which create the transform: plot, surface, contour
  (unless ``overplot'' is set), map\_set.

  Some commands which assume the existence of the coordinate transfer
  function: oplot, plots, contour (if ``overplot'' is set),
  map\_continents, map\_grid.


  \bi
    \item 2-d plots
   
	\bi 
	  \item plot

	    plot,Y -- Plots Y against it's index

	    plot,X,Y -- plots Y against X.
	    
	    Each call to plot clears the current graphics window
            \footnote{unless the system variable !p.multi[2] or !p.multi[3]  is
            non-zero. If so, the graphics window is cleared whenever
            !p.multi[0] = 0. !p.multi is the way you put multiple
             plots on the same page or screen}

	  \item oplot 

            A previous call to some graphics routine (e.g. ``plot''
            which establishes the graphics state of the current window
            is required before ``oplot'' can be called. And, since
            ``oplot'' uses the current environment, it's entirely
            possible that this command will result in \prbf{no data}
            being drawn to the 
	    screen. (Why?)

	      oplot,y -- Overplots Y against it's index in the
              current graphics window. 

	      oplot,x,y -- Overplots Y against X in the current graphics window. 

          \item plots

	    Like ``oplot'' this requires a previous call to some
            graphics routine that sets the graphics environment. 

	    plots,X,Y -- Plot Y versus X

	  \item variables pertaining thereto

	     psym: The symbol to use
             symsize: How big to make that symbol
             linestyle: What sort of line to use.
             Xrange: limit the X range of the data (plot only)
             Xrange: limit the Y range of the data (plot only)
	     Xticks/Yticks/XtickV/YtickV/Xtickformat/Ytickformat
            \ldots Hell, just look it up in IDLHELP!


	\ei
    \item 3-d plots
       \bi

         \item surface

	   Allows plotting of surface (i.e. 2-d) variables.

	   surface,Z -- Plots the 2-d array Z against it's indices.

	   surface,Z,X,Y -- plots Z against X and Y
	  
	     Either X and Y must have the same dimensionality as Z, or
            X must have as many rows as Z has columns and Y as many
            rows as Z has rows. (you figure it out!)

         \item contour

           Contouring of data, the flattened version of surface.

	   Contour,Z -- Make contours of Z against indices.
          
           Contour,Z,X,Y  -- Same as above, but use coordinate arrays
            X and Y. The same restrictions apply here as with ``Surface''

         \item shade\_surf 

	   Same as ``Surface'', but you can supply a ``shading'' to
            the surface, which is effectively a third dimension.

	  \item variables pertaining thereto

	   t3d, save, xrange, yrange, zrange

 	   \bi
	     \item Surface
	        
	       az,ax,ay -- the 'aspect' or rotation with respect to
               these axes.
	
	       
	     \item Contour

	       overplot, noerase, Z, fill, cell\_fill, follow, levels,
               nlevels, c\_colors, bilinear

	     \item Shade\_Surf

	       shades

	   \ei
       \ei

    \item Images
       \bi
         \item TV

	   TV is the method you put an image on a window.

	   IDL{TV,image,x,y}

            Places ``image'' at the device location
            ``x,y'' ``Device location is ``pixel number'' so if you
            say 

           \IDL{TV,image,100,100 }

            then the lower left hand corner of
            ``image'' will reside on the pixel which is 100 up and 100
            over from the lower-left hand corner of the window. And
            the array will be arranged on the window with first row
            bottom-most.

 	   TV is very stupid (or trusting, depending on your
            prespective) so if you try to TV a 500 by 500 element
            image into a 250 by 250 element window, 3/4 or your image
            will be off in la-la land.
	
	    It also does \prbf{nothing} about scaling the image to
            meet the demands of the graphics environment, so if you TV
            an integer array into a window expecting bytes, it will
            simply convert the image ``mod 256''.  One may do true
            color images in IDL, but by my assumptions, we're only
            doing 8 bit color, so the window will only be expecting
            byte arrays. \footnote{ See section \ref{sec:truecolor}
            for how to do true color images} 

	    These considerations mean that one should be careful to
            \prit{scale} images to the range of one byte, which brings
            us to our next routine:
       

         \item TVSCL

	    Which scales the image to the range of a byte. Personally
            I don't every use this routine, I prefer to control the
            scaling of images myself using the IDL builtin
            \prit{bytscl}.


       \ei
  \ei

  \item Calling syntax
  
   There are two types of programs in IDL: Procedures and
  Functions. Prodecudes are like subroutines in Fortran (without the
  'Call' syntax) and functions are, well  \ldots like functions.

   All programs, subprograms, procedures and functions have two types
  of thing that can be passed them: \prit{positional parameters} and
  \prit{keywords}. Positional parameters are just like the sort of
  parameter passing in ``C'' and ``Fortran'' and just about every
  other language you've ever seen, the first one in the call list is
  moved into the first one in the declaration list in the source file
  which defines the program, i.e. they line up one-to-one. Keywords
  have the calling syntax \prit{keyword = value} and can be placed in
  any order, because they are recognized by their name.

   The syntax is:

  \bi

   \item Procedure:

   \IDL{foo, a,b,c,/k1,k2=bar,k3=baz}

    Here I've called a procedure, which passes three positional
    parameters (a, b and c, in that order) and three keywords, k1, k2
    and k3.
  
    K1 is slightly different, but this is just a semantic sugar meaning
    exactly the same as ``k1=1.''  This sort of syntax is good for
    boolean quantities

    Notice I've said \prbf{absolutely nothing} about whether these
    quantities are input or output! And with the exception of the ``k1''
    keyword, which can only be input, it's because it doesn't
    matter. These quantities can be input, output, or both.

     Now, if I had, instead, said:

    \IDL{foo, 1,2,3,/k1,k2=3,k3='baz'}

    Then it's clear all of these quantities are \prbf{input} only, since
    IDL can't store an output quantity into a constant.

    In fact, if you tried this and the routine did try to return a value
    into the positional parameter ``a'', you'd get a run time error.

  \item Functions 
  
   Functions return values, it's a simple as that. So when you call
    them, you equate some variable to their output.  

  \IDL{bar=foo(a,b,/k1,k2='baz',k3=3*sin(indgen(10)\#barf))}

   Here, a and b are positional parameters, ``k1'' is one of those
  funny short hand keywords meaning k1=1, k2 is set to a constant
  ``baz'' and k3 is set to the return of some complicated expression
  involving all sorts of things.

  Clearly, only ``a'' and ``b'' \prit{may} return values, but ``bar''
  \prbf{must}!




\ei
	  

\newpage
\section{Some Basics, more explicitly}

  IDL is a \prbf{vector} based language, which is a generic term
  comprising arrays of all sorts up to eight dimensions. This means
  that all the standard operations take vectors as their arguments. At
  this level there's no difference between a \prit{vector}, a
  \prit{multi-dimensioned array} and \prit{scalar}. All operate in the
  same way.\footnote{There are, however, restrictions imposed on operations
  that arise from the relative sizes of the various arrays
  involved. More on this later}

  The fact that IDL is a vector based language means that you can do
  \prbf{element-by-element} operations on the entire array without
  having to \prbf{explicitly} iterate over the array, as you would in
  a compiled language such as ``C'' or ``Fortran.'' The user should
  reflect on the implications of these statements; it means that
  there is an implied loop \prbf{in the interpreter} itself.

  It is also a dynamically scoped language. The variables are created
  on the fly by assigning a pre-existing or created value to them. The
  type of the assigned to variable is that of the highest precision
  occuring in the expression to the right of the equals the sign. Only
  a variable may occur to the left of an equals sign.\footnote{like
  many languages, the left side must be an \prit{lvalue}} Hence there
  is very little need to \prit{predefine} variables; in fact it is
  discouraged that you do so unless absolutely necessary, since all
  you're doing is wasing space.


  So, a scalar may be created merely by assigning to it.

  \IDL{a=1} 

  creates a scalar short (16 bit) integer ``a'' whose value is ``1''.

  If you had, instead entered

  \IDL{a=1L}

  you would have created a 32 byte ``long'' value, and 

  \IDL{a=1.0} 

  would create a 4 byte float.
 
  One may create a vector or an array by simply enclosing a comma
  separated list of numbers in square brackets, as in:

  
  \IDL{a=[1,-1,3,5]}

  The type of the array will be the type of the highest precision
  constant appearing in the list. So:

  \IDL{a=[1b,2,4L,5.0d]} 

   will be a double array since the highest precision variable is a
   double, i.e. the ``5.0d.''

  You may even create arrays of multiple dimensions explicitly, by
   enclosing lists of lists, as in:

  \IDL{a=[ [1,2,3],[4,5,6] ]}
 
  I won't tell you want shape this array is, try to guess it! Then
   execute this at the IDL commandline and do

  \IDL{help,a} 

  to find out what it's dimensions are!


  While IDL will create an array having the highest possible
  precision, which means you can mix \prbf{numbers} in arrays 
  without regard to the consequences, you may not mix all types in a
  particular array. IDL arrays are either numeric, string, structure,
  object or pointer, and each must be kept separate. So you can't do:
 
  \IDL{a=[1b,``b'',2,3]}


  without causing an error. 

  Strangely enough, IDL will create this array while emitting an error
  (go ahead, try it!), but the second element, the ``b'', will be zero.

  In this regard, IDL is different from newer languages, like ``Perl,''
  which allow for arrays of completely hetrogeneous
  types.\footnote{And I wish IDL would follow Perl's lead to some
  extent, particularly with respect to hashes}

  You may, however, have arrays of structures; but here you aren't
  really violating the prohibition against mixing types within the
  array, now are you?


  Or, one may assign ``a'' as the result of an operation. For
  instance, one could do:

  \IDL{a=findgen(4)*2}\footnote{See below for a discussion of this function}

  Now a=[0,2,4,6]

  If you create either version of ``a'' variable above and then do 

  \IDL{help} 

  You'll see 

  \begin{verbatim}

	% At  $MAIN$                 
	A               INT       = Array[4]
	Compiled Procedures:
	    $MAIN$

	Compiled Functions:

   \end{verbatim}

  The first line \begin{verbatim} % At $MAIN$ \end{verbatim} show the
  program level you're in. In this case, since I executed the statment
  at a command prompt, I'm at the ``MAIN'' level.  If this command had
  been executed from within a procedure or function, this line would
  have reflected the fact that the command was not executed at the
  ``MAIN'' level but at the level of the calling procedure or
  function.\footnote{For instance: If you'd run this from a routine
  named \prit{foo}, this would read ``At \$FOO\$''}

  Then you see a list of the variable that exist at that moment, in
  our case only ``a,'' which is shown as a 4 elements short Integer
  array (or, vector as I call them)\footnote{see Section \ref{sec:types} below}

  The two lines which begin ``Compiled''\ldots would list the compiled
  procedures\footnote{Called ``subroutines'' in IDL} and functions, if
  there had been any when I executed this command.


  Now, there are many builtin IDL functions that will create arrays for
  you, \footnote{ See section \ref{sec:types} below}

  For instance, to create a sine curve over one period, do the
  following.

  \IDL{x=findgen(360*!pi/180)}

  \IDL{y=sin(x)}

  \IDL{plot,x,y}

   this will create a window with one period of a sine wave of unit
   period and amplitude.


\section{Getting Help}

  At the command prompt, do:

  \SH{idlhelp \&}

  This will open a window entitled ``IDL x.y Online Help'' The fastest
  way to get help on something is to search for it. Click on the
  ``Index'' button, type some word in the box, as you type the scroll
  box will incrementally go down the list finding those things that
  match your query. When you've found the item in that list, hit
  ``Display.'' Sometimes it will display a second list from which
  you'll have to make another choice.

  These are the ``man'' pages for IDL.

\newpage
\section{Variable Types}\label{sec:types}

   The \prit{size} builtin function provides information about the
   type of variable passed it.  For instance, if I do the following on
   the ``a'' I created above:

  \IDL{b=size(a)}

  \IDL{print,b}

   Which will print:

  \IDLOUT{   1           4           4           4 }

   ``Size'' returns information about the variable. Some of the output
   from IDLHELP on this function reads:




    \prit{The returned vector is always of longword type. The first element
    is equal to the number of dimensions of Expression. This value is
    zero if Expression is scalar or undefined. The next elements
    contain the size of each dimension, one element per dimension
    (none if Expression is scalar or undefined). After the dimension
    sizes, the last two elements contain the type code (zero if
    undefined) and the number of elements in Expression,
    respectively. The type codes are listed below under IDL Type
    Codes.}


   You can also pull out specific information using ``Size,'' for
   instance, doing:

   \IDL{b=size(a,/dimensions)}

   Will pull out the values of the various dimensions. In this case, 

  \IDL{print,size(a,/dim)} yeilds

  \IDLOUT{4}

    Since there's only one dimension, i.e. this is a vector of 4
    elements. If, you had done:

  \IDL{print,size(findgen(3,4),/dim)}

  You would have gotten:

  \IDLOUT{ 3 4 }




   
   The explicit IDL variable types are:

    
   \begin{center}
   \begin{tabular}{l|l|l|l}\hline
   \multicolumn{1}{|c|}{\sl Name} &
   \multicolumn{2}{|c|}{\sl Size} &
   \multicolumn{3}{|c|}{\sl Array} &
   \multicolumn{4}{|c|}{\sl Cast} \\
	Byte: & unsigned 1 byte character & bytarr() & byte() \\
        Int: &  signed 2 byte integer & intarr() & fix() \\
        Long: & signed 4 byte integer & lonarr() & long() \\
        Float: & 4 byte float & fltarr() & float() \\
	Double: & 8 byte float & dblarr() & double() \\
	Complex: & 2 4byte floats & complexarr() & complex()\\
	Double Complex: & 2 8 byte floats & dcomplexarr & complex()\\
	String: & A string of arbitrary length & strarr() & string()\\
	\hline
	Structure: & a structure & & & \\
	Pointer: & A pointer to a variable of any type&ptrarr() &&\\
	Object: & An object & objarr() & & \\
	\hline
	Unsigned Int: & unsigned 2 byte integer & uintarr() & uint()\\
	Unsigned Long: & unsigned 4 byte integer & ulonarr() & ulong() \\
	Long Long: &64 bit signed integer & lon64arr() & long64() \\
	ULongLong: & 64 bit unsigned integer & ulon64arr() & ulong64()\\
	
   \end{tabular}
   \end{center}

   Each of the numeric types also have an analog to create a ``range''
   of values in that type. Each of these will create an array of the
   specified number of elements starting at zero.

   \begin{center}
   \begin{tabular}{l|l|l|l}\hline
   \multicolumn{1}{|c|}{\sl Type} &
   \multicolumn{3}{|c|}{\sl Index Generator} \\
	Byte:    & bindgen() \\
        Int:     & indgen()/uindgen() \\
        Long:    & lindgen()/ulindgen \\
	long64:  & l64indgen()/ul64indgen \\
        Float:   & findgen() \\
	Double:  & dindgen() \\
	Complex: & cindgen() \\
	Double Complex: & dcindgen() \\
	String: & singden() \\ 
   \end{tabular}
   \end{center}

  An immediate consequence of these generators is that you can quickly
  and easily create cyclic functions. For instance:

  \IDL{y=sin(findgen(360*3.14159/180.))} 

  gives one period of a sine wave of period 1.

  
\section{Using}
  
  The great strength of IDL is that it allows you to manipulate data
  at an interactive prompt or run programs that range in complexity
  from the most simple to the fully developed large scale programs we
  normally associated with compiled languaged. There are some
  differences between the two methods of running, which I'll now
  discuss.


\subsection{Interactively}
\subsection{Programatically}
\subsection{I/O}
\subsubsection{Data Files}
\bi
  \item Text
  \item Binary
   \bi
     \item Flat Files
     \item HDF Files
   \ei
\ei
\subsubsection{Save Files}
\section{Advice}
\section{Tricks}
\end{document}
